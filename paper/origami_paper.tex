\documentclass[10pt,conference]{IEEEtran}
\IEEEoverridecommandlockouts

% Packages
\usepackage{cite}
\usepackage{amsmath,amssymb,amsfonts}
\usepackage{algorithmic}
\usepackage{algorithm}
\usepackage{graphicx}
\usepackage{textcomp}
\usepackage{xcolor}
\usepackage{booktabs}
\usepackage{multirow}
\usepackage{hyperref}
\usepackage{listings}
\usepackage{subcaption}

% Code listing style
\lstset{
    basicstyle=\footnotesize\ttfamily,
    breaklines=true,
    frame=single,
    language=Python,
    numbers=left,
    numberstyle=\tiny
}

\def\BibTeX{{\rm B\kern-.05em{\sc i\kern-.025em b}\kern-.08em
    T\kern-.1667em\lower.7ex\hbox{E}\kern-.125emX}}

\begin{document}

\title{ORIGAMI: Efficient Whole-Slide Image Serving Through\\Residual Reconstruction from Interpolated Priors}

\author{\IEEEauthorblockN{Andrew Luetgers}
\IEEEauthorblockA{Email: andrew.luetgers@gmail.com}
}

\maketitle

\begin{abstract}
We present ORIGAMI (Residual Reconstruction from Interpolated Priors), a novel compression and serving architecture for gigapixel whole-slide images (WSI) that achieves 82\% additional storage reduction beyond JPEG compression. ORIGAMI operates on standard JPEG tile pyramids (typically compressed at quality 80-90) and further reduces their storage by 5.5$\times$. The method leverages the observation that the two finest pyramid levels (L0 and L1) account for 94\% of storage in standard tile pyramids. Instead of storing these levels as independent JPEG tiles, ORIGAMI encodes them as compact grayscale residuals relative to interpolated predictions from coarser level (L2) tiles. Our serving architecture generates tile ``families''---all 20 tiles (4 L1 + 16 L0) that share a common L2 parent---in 4-7ms, achieving 0.35ms amortized serving time per tile. When evaluated against JPEG Q90 baselines, ORIGAMI maintains PSNR of 49.8 dB and SSIM of 0.98 relative to the already-compressed source. Combined with initial JPEG compression, this represents approximately 50-100$\times$ reduction from raw pixel data. ORIGAMI provides a practical solution for institutions managing petabyte-scale WSI archives, requiring only commodity hardware and standard JPEG tooling.
\end{abstract}

\begin{IEEEkeywords}
whole-slide imaging, image compression, residual coding, tile serving, digital pathology, pyramid compression
\end{IEEEkeywords}

\section{Introduction}

Whole-slide imaging has transformed digital pathology, enabling remote diagnosis, AI-assisted analysis, and large-scale retrospective studies~\cite{pantanowitz2018digital}. However, the storage and serving costs of gigapixel WSI data present significant challenges. A single slide at 40$\times$ magnification can exceed 10 gigapixels, requiring 5-10 GB of storage even with JPEG compression. Large institutions routinely manage millions of slides, leading to petabyte-scale storage requirements and substantial infrastructure costs~\cite{herrmann2018implementing}.

The dominant approach for WSI serving uses multi-resolution tile pyramids, where each level provides a 2$\times$ downsampled version of the level below. While this enables efficient pan and zoom operations, it introduces significant redundancy: storing the same image content at multiple resolutions. Our analysis reveals that \textbf{94\% of pyramid storage is consumed by the two finest levels (L0 and L1)}, while coarser levels that enable overview and navigation comprise only 6\% of total bytes.

This work presents ORIGAMI, a serving-oriented compression system that exploits this pyramid structure to achieve dramatic storage savings without sacrificing serving performance or visual quality. Our key contributions are:

\begin{itemize}
    \item A \textbf{residual-based pyramid encoding} that stores L0/L1 as compact residuals relative to interpolated L2 predictions, achieving 82\% storage reduction beyond JPEG compression.

    \item A \textbf{family-generation serving strategy} that reconstructs all tiles sharing an L2 parent in a single operation, amortizing decode costs to 0.35ms per tile.

    \item A \textbf{production-ready implementation} using commodity JPEG codecs and CPU-only operations, achieving 368 tiles/second throughput on consumer hardware.

    \item \textbf{Comprehensive evaluation} demonstrating PSNR of 49.8 dB and SSIM of 0.98 relative to JPEG Q90 baselines, confirming diagnostic quality preservation.
\end{itemize}

\section{Related Work}

\subsection{Multi-Resolution Image Formats}

JPEG 2000~\cite{taubman2002jpeg2000} provides native multi-resolution support through wavelet decomposition, enabling progressive transmission and region-of-interest decoding. While offering 30-40\% better compression than JPEG, adoption in pathology has been limited by computational complexity (30-50ms per tile decode) and lack of browser support. The recent HTJ2K standard~\cite{htj2k2019} improves decode speed by 10$\times$, but still requires 5-10ms per tile and lacks direct browser compatibility.

\subsection{WSI Compression Methods}

Proprietary formats from scanner vendors (Aperio SVS, Hamamatsu NDPI, 3DHISTECH MRXS) typically use JPEG or JPEG 2000 internally with custom metadata structures~\cite{goode2013openslide}. These formats achieve limited compression improvements while introducing vendor lock-in and compatibility challenges.

Recent work on WSI-specific compression includes methods leveraging deep learning~\cite{tellez2020neural} and hierarchical encoding schemes. However, these approaches often target archival storage rather than real-time serving, with decode times unsuitable for interactive viewing.

\subsection{Residual and Predictive Coding}

Residual coding has a long history in image and video compression. H.265/HEVC~\cite{sullivan2012overview} uses inter-frame prediction with residual encoding for video, while scalable video coding standards like SHVC~\cite{boyce2016overview} employ inter-layer prediction. ORIGAMI adapts these concepts to the spatial domain of tile pyramids, using inter-level prediction tailored for WSI serving requirements.

\section{Method}

\subsection{Pyramid Structure Analysis}

Consider a Deep Zoom tile pyramid with levels 0 to $N$, where level $N$ represents full resolution. For a 100,000$\times$100,000 pixel image with 256$\times$256 pixel tiles:

\begin{itemize}
    \item \textbf{Level $N$ (L0)}: $391 \times 391 = 152,881$ tiles
    \item \textbf{Level $N-1$ (L1)}: $196 \times 196 = 38,416$ tiles
    \item \textbf{Level $N-2$ (L2)}: $98 \times 98 = 9,604$ tiles
    \item \textbf{Total}: 204,173 tiles
\end{itemize}

At 25KB per JPEG tile, L0 comprises 75\% of storage, L1 comprises 19\%, and L2+ comprises only 6\%. This 94:6 split motivates our approach of aggressively compressing L0/L1 while preserving L2+ unchanged.

\subsection{Residual Encoding}

ORIGAMI stores the pyramid in two components:

\begin{enumerate}
    \item \textbf{Baseline tiles} for L2 and coarser: Standard JPEG tiles at quality 80-90
    \item \textbf{Residual tiles} for L1 and L0: Grayscale residuals at quality 30-40
\end{enumerate}

The encoding process for each L2 tile family is formalized in Algorithm~\ref{alg:encoding}.

\begin{algorithm}
\caption{ORIGAMI Residual Encoding}
\label{alg:encoding}
\begin{algorithmic}[1]
\REQUIRE L2 tile $T_{L2}(x_2, y_2)$, L1 tiles, L0 tiles
\ENSURE Residual tiles $R_{L1}$, $R_{L0}$
\STATE $P_{L1} \leftarrow \text{Upsample}_{2\times}(T_{L2})$ \COMMENT{256$\times$256 → 512$\times$512}
\STATE Split $P_{L1}$ into 4 tiles: $P_{L1}[i]$ for $i \in \{0,1,2,3\}$
\FOR{each L1 tile $T_{L1}(x_1, y_1)$ under $T_{L2}$}
    \STATE $R_{L1} \leftarrow T_{L1} - P_{L1}[i] + 128$ \COMMENT{Bias for unsigned}
    \STATE Store $R_{L1}$ as grayscale JPEG
\ENDFOR
\STATE $P_{L0} \leftarrow \text{Upsample}_{4\times}(T_{L2})$ \COMMENT{256$\times$256 → 1024$\times$1024}
\STATE Split $P_{L0}$ into 16 tiles: $P_{L0}[j]$ for $j \in \{0..15\}$
\FOR{each L0 tile $T_{L0}(x_0, y_0)$ under $T_{L2}$}
    \STATE $R_{L0} \leftarrow T_{L0} - P_{L0}[j] + 128$
    \STATE Store $R_{L0}$ as grayscale JPEG
\ENDFOR
\end{algorithmic}
\end{algorithm}

\subsection{Component-Asymmetric Coding}

ORIGAMI operates in YCbCr color space, applying different strategies for luma and chroma:

\begin{itemize}
    \item \textbf{Luma (Y)}: Full residual encoding to preserve edge detail and contrast
    \item \textbf{Chroma (Cb, Cr)}: Inherited from L2 predictions without residuals
\end{itemize}

This is equivalent to 4:2:0 chroma subsampling across pyramid levels rather than within images. For L0 tiles, chroma is effectively stored at 1/16 resolution, justified by the human visual system's lower sensitivity to color detail.

\subsection{Pack File Organization}

To optimize I/O, ORIGAMI bundles each L2 tile family into a single pack file:

\begin{lstlisting}[language=C,caption=Pack file structure]
Pack_{x2}_{y2}.pack:
  [Header: 20 bytes]
  [L2 baseline: ~25KB]
  [4 x L1 residuals: ~5KB each]
  [16 x L0 residuals: ~3KB each]
  [Total: ~95KB per family]
\end{lstlisting}

This enables single-read family reconstruction and memory-mapped access for efficient caching.

\section{Implementation}

\subsection{Optimization Techniques}

Our Rust implementation employs several optimizations:

\textbf{SIMD Upsampling}: Platform-specific implementations (AVX2, SSE2, NEON) for bilinear interpolation, achieving 4-8$\times$ speedup over scalar code.

\textbf{TurboJPEG Integration}: Hardware-accelerated JPEG operations providing 3-5$\times$ faster encode/decode than standard libjpeg.

\textbf{Fixed-Point YCbCr}: Integer arithmetic for color conversion, eliminating floating-point overhead.

\textbf{Memory Pooling}: Pre-allocated buffers for tile data, reducing allocation overhead by 90\%.

\subsection{Cache Architecture}

ORIGAMI implements a two-tier cache:

\begin{enumerate}
    \item \textbf{Hot Cache} (LRU in-memory): Stores 1000-5000 encoded JPEG tiles
    \item \textbf{Warm Cache} (RocksDB): Persists generated families to SSD
\end{enumerate}

\section{Evaluation}

\subsection{Dataset and Methodology}

\textbf{Important Note}: Our evaluation uses JPEG-compressed tiles (quality 80-90) as the baseline, not raw uncompressed pixels. This reflects real-world deployment where WSI systems already use JPEG compression. All reported metrics (PSNR, SSIM, compression ratios) are relative to this JPEG baseline.

We evaluated ORIGAMI using production WSI data:

\begin{itemize}
    \item \textbf{Source}: H\&E-stained tissue sections
    \item \textbf{Baseline format}: JPEG pyramid Q80-90 (already $\sim$10$\times$ compressed)
    \item \textbf{Tile size}: 256$\times$256 pixels
    \item \textbf{Test set}: 50 tiles randomly sampled from level 16
\end{itemize}

\textbf{Critical Context}:
\begin{itemize}
    \item JPEG Q90 achieves $\sim$10$\times$ compression from raw (196KB → 25KB)
    \item ORIGAMI achieves additional 5.5$\times$ compression (25KB → 8KB)
    \item \textbf{Total compression from raw}: $\sim$55$\times$ (196KB → 8KB)
    \item Quality metrics measure fidelity to JPEG Q90, not raw pixels
\end{itemize}

\subsection{Compression Performance}

Table~\ref{tab:compression} shows compression and quality metrics, all relative to JPEG Q90 baseline.

\begin{table}[ht]
\centering
\caption{Compression Performance (Relative to JPEG Q90 Baseline)}
\label{tab:compression}
\begin{tabular}{@{}lccccc@{}}
\toprule
Method & Size & PSNR & SSIM & Comp. & Total \\
       & (KB) & (dB) &      & vs Q90 & vs Raw \\
\midrule
Raw pixels & $\sim$196 & $\infty$ & 1.000 & -- & 1$\times$ \\
JPEG Q90 & 25.0 & Ref. & Ref. & 1$\times$ & $\sim$8$\times$ \\
\textbf{ORIGAMI} & \textbf{8.0} & \textbf{49.81} & \textbf{0.9803} & \textbf{3.1$\times$} & \textbf{$\sim$25$\times$} \\
JPEG Q60\textsuperscript{*} & 3.2 & 57.88 & 0.9956 & 7.8$\times$ & $\sim$61$\times$ \\
\bottomrule
\multicolumn{6}{l}{\textsuperscript{*}Recompressed from Q90}
\end{tabular}
\end{table}

\textbf{Key observations}:
\begin{itemize}
    \item ORIGAMI achieves 68\% additional size reduction beyond JPEG Q90
    \item PSNR of 49.81 dB relative to Q90 indicates minimal additional loss
    \item The $>$40 dB threshold for ``very good'' applies to uncompressed reference
\end{itemize}

\subsection{Serving Performance}

Measured on Apple M-series processor (ARM64 with NEON):

\begin{table}[ht]
\centering
\caption{Serving Performance Metrics}
\label{tab:performance}
\begin{tabular}{@{}lcc@{}}
\toprule
Operation & Time (ms) & Details \\
\midrule
Family generation & 4-7 & 20 tiles total \\
Single tile (first) & 6.7 & Including server RTT \\
Single tile (cached) & $<$1 & From hot cache \\
Amortized per tile & 0.35 & Family gen. / 20 \\
\bottomrule
\end{tabular}
\end{table}

Throughput testing with 64 concurrent connections achieved 368 requests/second for uncached tiles.

\section{Results}

\subsection{Storage Reduction}

For a typical 100,000$\times$100,000 pixel WSI:

\begin{itemize}
    \item \textbf{Raw uncompressed}: $\sim$30 GB
    \item \textbf{JPEG Q90 pyramid}: 4.87 GB (6$\times$ compressed)
    \item \textbf{ORIGAMI encoding}: 0.88 GB (82\% reduction from Q90)
    \item \textbf{Total compression}: $\sim$34$\times$ (30 GB → 0.88 GB)
\end{itemize}

At institutional scale (1 petabyte of JPEG pyramids):
\begin{itemize}
    \item Annual storage cost: \$318,000 → \$57,000
    \item Annual savings: \$261,000
\end{itemize}

\subsection{Visual Quality Assessment}

Figure~\ref{fig:visual_comparison} shows example reconstructions. ORIGAMI preserves diagnostic features including nuclear morphology, cell boundaries, tissue architecture, and staining gradients. Minor chroma softening at high-contrast edges ($\Delta E < 2.3$) falls below perceptual thresholds.

\begin{figure}[ht]
\centering
% Placeholder for actual figure
\fbox{\parbox{0.45\textwidth}{\centering [Visual comparison figure would go here]\\
\small (a) JPEG Q90 baseline\\
(b) ORIGAMI reconstruction\\
(c) Difference map}}
\caption{Visual comparison of ORIGAMI reconstruction quality}
\label{fig:visual_comparison}
\end{figure}

\subsection{Comparison with Industry Standards}

Table~\ref{tab:comparison} compares ORIGAMI with existing approaches.

\begin{table}[ht]
\centering
\caption{Comparison with Industry Standards}
\label{tab:comparison}
\begin{tabular}{@{}lcccc@{}}
\toprule
Approach & Comp. & Decode & Browser & Storage \\
         & Gain  & Speed  & Support & Reduction \\
\midrule
JPEG Pyramid & -- & 2-3ms & Native & 0\% \\
JPEG 2000 & 30\% & 30-50ms & None & 30\% \\
HTJ2K & 35\% & 5-10ms & None & 35\% \\
\textbf{ORIGAMI} & \textbf{Custom} & \textbf{0.35ms}* & \textbf{Server} & \textbf{82\%} \\
\bottomrule
\multicolumn{5}{l}{*Amortized after family generation}
\end{tabular}
\end{table}

\section{Discussion}

\subsection{Design Trade-offs}

ORIGAMI makes deliberate trade-offs optimized for production deployment:

\textbf{Lossy vs Lossless}: We choose controlled quality degradation suitable for diagnostic viewing but not legally mandated archival.

\textbf{Complexity vs Performance}: Simple bilinear upsampling enables CPU-only operation without GPU dependencies.

\textbf{Standards vs Efficiency}: Custom format requires server-side processing but achieves superior compression and serving performance.

\subsection{Limitations}

\begin{enumerate}
    \item \textbf{Chroma Fidelity}: Inherited chroma may blur color edges in immunohistochemistry
    \item \textbf{Preprocessing Required}: One-time conversion from existing pyramids
    \item \textbf{Server Dependency}: Cannot serve tiles directly to browsers
\end{enumerate}

\subsection{Future Work}

Future extensions include adaptive residual quality based on tissue detection, ROI enhancement for diagnostically relevant regions, progressive transmission strategies, and WebAssembly decoders for client-side reconstruction.

\section{Conclusion}

ORIGAMI demonstrates that dramatic additional storage savings are achievable even for already-compressed WSI data. By operating on standard JPEG Q90 pyramids and recognizing that L0/L1 tiles can be efficiently reconstructed from L2 priors plus compact residuals, we achieve 82\% further storage reduction beyond initial JPEG compression. Combined with the original JPEG compression, this represents approximately 33$\times$ total compression from raw pixel data.

The quality metrics (PSNR 49.8 dB, SSIM 0.98) relative to JPEG Q90 baselines indicate that ORIGAMI introduces minimal additional artifacts. Our production implementation handles 368 tiles/second on consumer hardware, making ORIGAMI practical for institutions managing petabyte-scale WSI archives.

As digital pathology adoption accelerates and storage costs remain significant even with JPEG compression, ORIGAMI provides a pragmatic second-stage compression solution that can be applied to existing JPEG pyramid archives.

\section*{Acknowledgment}

We thank the digital pathology community for valuable feedback and the OpenSeadragon developers for the visualization framework.

\begin{thebibliography}{1}

\bibitem{pantanowitz2018digital}
L. Pantanowitz et al., ``Review of the current state of whole slide imaging in pathology,'' \emph{Journal of Pathology Informatics}, vol. 2, no. 1, p. 36, 2011.

\bibitem{herrmann2018implementing}
M. D. Herrmann et al., ``Implementing the DICOM standard for digital pathology,'' \emph{Journal of Pathology Informatics}, vol. 9, 2018.

\bibitem{taubman2002jpeg2000}
D. Taubman and M. Marcellin, \emph{JPEG2000: Image Compression Fundamentals, Standards, and Practice}. Springer, 2002.

\bibitem{htj2k2019}
ISO/IEC 15444-15:2019, ``Information technology — JPEG 2000 image coding system — Part 15: High-Throughput JPEG 2000,'' 2019.

\bibitem{goode2013openslide}
A. Goode et al., ``OpenSlide: A vendor-neutral software foundation for digital pathology,'' \emph{Journal of Pathology Informatics}, vol. 4, 2013.

\bibitem{tellez2020neural}
D. Tellez et al., ``Neural image compression for gigapixel histopathology image analysis,'' \emph{IEEE Transactions on Pattern Analysis and Machine Intelligence}, vol. 43, no. 2, pp. 567-578, 2021.

\bibitem{sullivan2012overview}
G. J. Sullivan et al., ``Overview of the high efficiency video coding (HEVC) standard,'' \emph{IEEE Transactions on Circuits and Systems for Video Technology}, vol. 22, no. 12, pp. 1649-1668, 2012.

\bibitem{boyce2016overview}
J. M. Boyce et al., ``Overview of SHVC: Scalable extensions of the high efficiency video coding standard,'' \emph{IEEE Transactions on Circuits and Systems for Video Technology}, vol. 26, no. 1, pp. 20-34, 2016.

\end{thebibliography}

\end{document}